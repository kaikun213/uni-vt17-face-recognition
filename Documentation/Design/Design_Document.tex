\documentclass[a4paper,11pt]{article}

% ------------ Renew commands to have correct counting in enumerate env

\renewcommand{\theenumi}{\thesubsection.\arabic{enumi}}
\renewcommand{\labelenumii}{\theenumii}
\renewcommand{\theenumii}{\theenumi.\arabic{enumii}.}

% ------------ Import Requirements Elicitation for Referencing
\usepackage{xr}
\externaldocument{../Requirements/Requirements_Elicitation}

\usepackage[utf8]{inputenc}
\usepackage[top=3cm, bottom=3cm, left=3cm, right=3cm]{geometry}
\usepackage[none]{hyphenat}
\usepackage{xcolor,colortbl}
\definecolor{Gray}{rgb}{0.9,0.9,0.9}

% ------------ Title section
\title{
\vspace{-8cm}
\begin{flushleft}
    \vspace{10cm}
    \normalfont \normalsize
    %\Huge Bachelor/Master Thesis Project \\
    \vspace{-1.3cm}
\end{flushleft}
\vspace{3cm}
\begin{flushleft}
    \huge Face Recognition System \\
    \LARGE  Design Document\\
\end{flushleft}
\null
\vfill
\begin{minipage}{\textwidth}
\begin{flushleft} \large
\emph{Authors:} Walid Balegh, Jakob Heyder, Sarpreet Singh Buttar, Henry \hspace{45pt} Pap and Oscar Maris \\ % Author
%\emph{Supervisor:} Name of your supervisor\\ % Supervisor
%\emph{Examiner:} Dr.~Mark \textsc{Brown}\\ % Examiner (course manager)
\emph{Semester:} VT 2017\\ %
%\emph{Subject:} Computer Science\\ % Subject area
\end{flushleft}
\end{minipage}
}

\date{}

\begin{document}

\maketitle

\newpage

\tableofcontents

\newpage


\section{Introduction}

\subsection{Purpose}

\subsection{Priorities}
%Priorities
%	- Security - Encryption and Authentication , sensible data
%	- Portability/Integration \& Usability - clear defined API
%	- Reliability

\subsection{Overview}

\section{Major Design Issues}
In this section all major design issues and decisions will be discussed. Rational and alternatives will be presented and the detail of evaluating the alternatives depends on the trade off the decision has.

\section{Architectural Design}
As desired from the customer the system will have a Client-Server structure. Where the in section \ref{ReqDefinitions} of the Requirements Elicitation defined UAM and AAM are the client modules and the URM and ARM are the server modules. Further we may reference Server or Client in general if  the reference is to both components - User and Admin module.

\subsection{Languages \& Frameworks}
\subsubsection{Server-side}
For the Server the \textbf{Java programming language} is one of the most used. Especially REST functionality is supported by various frameworks with large communities. Rationals which led to the choice are listed below.
\begin{itemize}
\item Great flexibility \& portability due to platform independence with the JVM.
\item High productivity because of existing frameworks and solutions
\item Good support by a large community of developers
\end{itemize}
\textit{Concerns are that it is more difficult for multi-threaded development and reactive programming. The scalability can still be guaranteed because ... => frameworks}

The Spring Framework will be used on the server side of the system. More detailed the Spring Web, Security, Data and Boot modules.
The Spring Framework allows fast, enterprise scale development of applications by providing features for security, RESTful applications and Data Management. Rationals for using the different modules are listed below.
\begin{itemize}
\item Great Portability and Integration - it is supported by various cloud providers to make deployment and continuous development possible.
\item Spring Boot provides an embedded application server which allows fast and easy setup of an application.
\item Configurability - Spring is easily set up and gives good default solutions but also provides the possibility to configure the details to the application needs
\item The Spring framework provides RESTful support which is asked for from the customer and fills the application needs.
\item Spring Data provides a convenient way to implement CRUD functionality for accessing and modifying data. It supports various Database technologies such as JPA and generates boilerplate code at run time which reduces developing costs.
\item Spring Security provides enterprise ready security features for authentication and encryption without much setup.
\end{itemize}

\subsubsection{Client-side}


\noindent The client application has to be working on iOS, Android, Microsoft
phone and WEB, there is a ton of Cross Platform Mobile development tools
out there.

\noindent For example

\begin{itemize}
\item
  \textbf{Xamarin} - which is the most popular choice, a free trail is
  available and it use the language C\#. This will make it more
  structured as C\# is an \emph{OO} language.
\item
  \textbf{Phone Gap} - which is the most well known tool, it is open
  source meaning that it is free. It uses the common web languages to
  create hybrid apps i.e.~HTML, CSS, JavaScript.
\item
  \textbf{appcelerator} - lets developers use JavaScript to build their
  apps, provide mobile testing, it has a GUI to create design (which
  uses common HTML and CSS), and lastly it is free.
\end{itemize}

\noindent There is plenty more but there is one last option that we want to
consider which is to create it only to WEB, the question we need to ask
ourselves is: \emph{are we going to produce a app for the different app
stores or do we only want it to work on the different devices (Android,
iOS, Microsoft phone)?} We will not create an app for the different app
stores out there since that is not in the requirements, because of this
the application will be built using common WEB languages. We will use
responsive design to style our app so it will be desktop, tablet and
mobile friendly. For this we will use a CSS framework, all CSS
frameworks comes also with a JavaScript framework for design purpose
(animation etc.).

\noindent CSS frameworks:

\begin{itemize}
\item
  \textbf{Bootstrap} - the most common framework out there, easy to use
  and creates fast design. Great for dynamic designing thanks to its
  grid system. The major drawback is that it will look boring and old.
\item
  \textbf{Material Framework} - Google's own framework, a google look
  alike framework.
\item
  \textbf{Semantic UI} - a fresh framework that has grown quite popular
  in the last couple of years. It uses the JavaScript jQuery framework
  which is easy to use and has great AJAX calls which can be helpful.
\end{itemize}

\noindent This is only a design option and we will go with the Semantic UI because
it has a suiting design, and for the use of jQuery.

\subparagraph{To summarize}\label{to-summarize}
We will use basic WEB development for our application as the requirement
only says that it should work on iOS, Android etc.\\
The framework that will be in used for WEB development is Semantic UI
for CSS and jQuery for JavaScript as it comes with Semantic UI.

\subsection{Platforms \& Technologies}

\subsubsection{Communication}
The Communication will be over IP/TCP to have reliable transport and uses HTTPS on the application layer. This ensures general security by using SSL/TLS and ensures data integrity and privacy by authenticating the application.
\textit{=> Read up about SSL/TLS and HTTPS what does it secure how does it work?}

\subsubsection{Platform Server}
The Server will run on a cloud platform. This gives several advantages which are listed below. Especially easy setup and management are essential for this project during development and by using Java the components are platform independent which allows later changes during production.
\begin{itemize}
\item Easy to manage and setup - no System administrator needed
\item Allows fast and continuous development and testing
\item Cheap and scalable solution
\end{itemize}

\subsubsection{Platform Client}
The Client, more specific the in the to be developed UAM will be Web compatible. Since it is written in Javascript
Mobile Development = iOs/ Android / WEB
	- Market share
	- Cross platform frameworks
	- WEB can be accessed by every device with a browser

\subsection{External Services}

\subsubsection{Cloud Platform}
For the development in the cloud, Heroku is among AWS, Microsoft Azure, Google application engine and others a common choice. It supports good conditions for development and support frameworks for features such as database deployment. This gives a convenient way to get the system fastly up and working. Listed are features it provides.
\begin{itemize}
\item Native support for Java and Spring Boot application deployment
\item Addon support for rational Databases (e.g. MYSQL)
\item Github integration for continuous development
\item Free use for small scale applications (development)
\end{itemize}

\subsubsection{External Face Recognition API}
%\textit{Comparison of
%	- SkyBioMetric(Free), LambdaLabs(Best support, but costly), OpenFace(OpenSource, more work)}
%
There are some External Face Recognition API out there such as
\begin{itemize}
\item \emph{SkyBiometry: } It is a cloud-based face detection and recognition software which provides a high-precision biometric identification for over 20 years. In addition, it also provide API client libraries in various languages such as Java, C\#, Pyton etc for giving a quick start to the developers. Regarding the usage limits, it has a free subscription which allows 100 methods calls hourly and 5000 monthly. It provide SSL support and its API uses REST interface which means all the API methods are called over the Internet using standard HTTP methods and responses are generated in XML or JSON.
\item \emph{Lambda Labs: } It permits developers to send an image link to their service for the identification. In addition, it also allow to create an album of photos, analyze and compare new images with existing ones. Regarding the usage limits, it has a no free subscription and minimum cost is \$9/month. It does not provide SSL support and its API also uses REST interface and responses are generated in JSON.
\item \emph{OpenFace: } It is a open source web service which provide facial detection technologies. Its API uses REST interface and accept image from the developer and return a JSON response. Currently, it does not support SSL and can only detect up to 80 points on a given image.
\end{itemize}
We have found that \emph{OpenFace} does not provide sufficient functionality as compare to others wheres \emph{Lamba Labs} does provide needed functionality but with a cost of \$9/month. In result, \emph{SkyBiometry} is the free and suitable option for our project.




\subsection{Application specific}
Authentication will be done by providing a username and password for registered Users and Admins. The Registration will be exclusive over  a non automated channel by contacting the Customer/Developers to verify a service. The in the Requirements Elicitation mentioned credentials refer further to a user name and password.

\subsection{Database and data format}
The data will be formatted in standard JSON for communications between the server and the client. The format is human-readable and widely supported. It also supports the requirements of a RESTful application.
\newline
\newline
\noindent
The database will be MYSQL a rational database.
It is one of the most used rational databases and therefore provides sufficient features, support and scalability for the application. It is also compatible with the used Spring framework and the Heroku cloud platform. It validates data and ensures integrity.

\section{Architecture (Component Diagram)}

IDEAS ---------------------------------------------------

Client-Server architecture => Repositories


Used architectural patterns:
Layered Approach
	- Access Layer (Client/UI/API), Database layer (ORM), Data ..
Client/Server architecture
	- Separate development, Server provides API (cohesive service)
	- ~> Service oriented architecture (External Service, ASM, USM) supports continuous development, integration and is very scalable. JSON Communication as standard
MVC
	- Flexibility, Testability increased, separate components as Client/Server and separate logic and data etc.

END IDEAS ---------------------------------------------------

\section{Components - Static modeling (Class Diagrams)}

\section{Use cases - Behavioral modeling (Sequence Diagrams)}


\end{document}
